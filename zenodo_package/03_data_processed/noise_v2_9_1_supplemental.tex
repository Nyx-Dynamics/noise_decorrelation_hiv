% Supplementary Materials for Nature Communications
% "Noise Correlation Length Modulates Neuroprotection in Acute HIV"
%
%%%%%%%%%%%%%%%%%%%%%%%%%%%%%%%%%%%%%%%%%%%%%%%%%%%%%%%%%%%%%%%%%%%%%%

\documentclass[pdflatex,sn-mathphys-num]{sn-jnl}

%%%% Standard Packages
\usepackage{graphicx}%
\usepackage{multirow}%
\usepackage{amsmath,amssymb,amsfonts}%
\usepackage{amsthm}%
\usepackage[title]{appendix}%
\usepackage{xcolor}%
\usepackage{textcomp}%
\usepackage{booktabs}%
\usepackage{algorithm}%
\usepackage{algorithmicx}%
\usepackage{algpseudocode}%
\usepackage{listings}%
\usepackage{longtable}%
\usepackage{textgreek}%
\usepackage{lmodern}%
\usepackage{rsfso} % Use scalable script font to avoid missing font sizes
\usepackage[utf8]{inputenc}%
\usepackage{amssymb}%
\usepackage{newunicodechar}%
\newunicodechar{−}{-}


\raggedbottom

\begin{document}


\title[Supplementary Materials]{Supplementary Materials for:\\Noise Correlation Length Modulates Neuroprotection in Acute HIV: Evidence from Expanded Multi-Cohort Analysis of 296 Patients}

\author*[1]{\fnm{A.C.} \sur{Demidont, DO}}\email{acdemidont@nyxdynamics.org}

\affil*[1]{\orgname{Nyx Dynamics, LLC}, \orgaddress{\street{268 Post Road East}, \city{Fairfield}, \postcode{06428}, \state{Connecticut}, \country{USA}}}

\maketitle

\tableofcontents

\newpage

\section{Supplemental Methods: Expanded Dataset (3:1:1) — Noise Decorrelation HIV Models}
\begin{verbatim}
This document records the exact environment, commands, run IDs, diagnostics, and figure-generation steps used to reproduce the expanded-dataset analyses (3:1:1 primary) for the phenomenological (v3.6, v3) and forward mechanistic ODE (enzyme v3) models. It is written to ensure end-to-end reproducibility from a clean checkout on Apple Silicon (arm64) Macs.

{Environment (Apple Silicon, Conda-forge)}

- Hardware/OS: Apple Silicon (arm64) macOS
- Python: 3.12 (arm64)
- Distribution: Miniforge/Conda-forge

Create the environment:

```bash
brew install --cask miniforge
conda init zsh
exec zsh

conda create -n hiv-models -c conda-forge \
  python=3.12 numpy scipy pandas matplotlib pymc arviz pytensor \
  openblas llvm-openmp -y
conda activate hiv-models


Set threading for stable performance:

```bash
export OMP_NUM_THREADS=4
export OPENBLAS_NUM_THREADS=1

{Data and Era Definition}

- Primary expanded dataset (phase-balance 3:1:1) files live under:
  - {data/extracted\_expanded/data\_ratios\_comparison/bayesian\_inputs\_3\_1\_1.csv}
  - `data/extracted_expanded/data_ratios_comparison/enzyme_inputs_3_1_1.csv`

- Era classification (pre vs post modern ART):
  - `pre_modern`: year <= 2006
  - `post_modern`: year >= 2007
  - If year is missing, rows are labeled `unknown`.

- The loader attaches `art_era` via a fallback chain (measurement year → publication year → study→year mapping). In this project, a one-off preprocessing step filled missing `publication_year` for known studies (e.g., `Chang_2002` → 2002, `Young_2014` → 2014). This enabled pre/post runs.

Note: rows with `art_era = unknown` are included only in `--era both` analyses; they are excluded from `pre_modern`/`post_modern` splits to avoid imputing era when year information is unavailable.

{Models and Roles}

    
- v3.6 (phenomenological): primary Bayesian model used for manuscript inference and figures. Better posterior predictive fit vs v3.
- v3 (phenomenological): optional parity model; included for completeness in sensitivity or historical comparisons.
- enzyme v3 ODE (mechanistic forward): produces steady-state NAA/Cr and Cho/Cr by phase using the ξ → Π_ξ mapping (Option A), with ART era treated as a covariate for filtering/stratification of inputs, not as a mechanistic modifier.

All Bayesian variants employ Option A mapping exclusively:
- Mechanistic driver: `xi_estimate_nm`
- Derived modulation: `Π_ξ = (xi / xi_baseline)^(−β_ξ)`
- `enzyme_activity_fold` columns are retained for validation only (not used in the forward modulation).

{Primary Converged Run (Both Eras): v3.6, 3:1:1}

- Command (example):

```bash
export OMP_NUM_THREADS=4 OPENBLAS_NUM_THREADS=1
python -m quantum.bayesian_v3_6_runner --ratio 3_1_1 --era both \
  --draws 1000 --tune 1000 --chains 4 --seed 2025
```

- Run ID used for figures: 
  - `results/bayesian_v3_6/3_1_1/both/20251123T223826Z_a8c794cc/`

- Diagnostics (from ArviZ):
  - max R-hat $\approx$ 1.0064 ($\leq$ 1.01)
  - min ESS bulk ≈ 1079
  - min ESS tail ≈ 1216
  - Divergences: none at final settings (or negligible after modest tuning)

- Key outputs used:
  - `trace_v3_6.nc` (InferenceData NetCDF)
  - `posterior_predictive.csv` (observed vs posterior predicted means)
  - `run_manifest.json` (per-run provenance)

- Manuscript figures generated (see Figure Generation section):
  - `forest_hdi_3_1_1_both_v3_6.png`
  - `posterior_beta_xi_kde_3_1_1_both_v3_6.png`
  - `posterior_xi_control_kde_3_1_1_both_v3_6.png`
  - `posterior_xi_acute_kde_3_1_1_both_v3_6.png`
  - `posterior_xi_chronic_kde_3_1_1_both_v3_6.png`
  - `ppd_overlay_3_1_1_both_v3_6.png`
  - `era_effect_violin_3_1_1_both_v3_6.png`


{Pre/Post Era Splits (Sensitivity Note)}

    
- After filling publication years by study, the dataset contained:
  - `pre_modern`: 1 row (Chang 2002)
  - `post_modern`: 27 rows
  - `unknown`: 2 rows

- With `pre_modern` only (N=1), era-specific inference is weakly identified. Short-tuning MCMC exhibited many divergences (as expected). Longer tuning and higher target acceptance (e.g., `--tune 6000`, `--target-accept 0.97`) mitigate but do not fully eliminate divergences due to the single observation.

- We therefore report the primary inference and all manuscript figures from the converged **both-eras** v3.6 run, with ART era treated as a covariate in the combined model. Pre/post panels may be shown as supplementary illustrations with appropriate caveats.

{Forward Mechanistic ODE (enzyme v3)}


- Command (3:1:1, both):

```bash
python -m quantum.run_enzyme_v3 --ratio 3_1_1 --era both
```

- Run ID used for ODE figure:
  - `results/enzyme_v3/3_1_1/both/20251123T200507Z_1bdc3701/`
- Figure produced (manuscript):
  - `ode_phase_ratios_3_1_1_both.png`

{Figure Generation (Manuscript Set)}

A dedicated generator produces the legacy-style figure set into `figures/figures/` based on the latest (or specified) run.

- Both eras (v3.6, 3:1:1):

```bash
python -m quantum.legacy_figures \
  --model bayesian_v3_6 --ratio 3_1_1 --era both \
  --run-id 20251123T223826Z_a8c794cc \
  --suffix 3_1_1_both_v3_6 --outdir figures/figures
```

- Pre / Post (optional supplemental panels):

```bash
python -m quantum.legacy_figures \
  --model bayesian_v3_6 --ratio 3_1_1 --era pre_modern \
  --suffix 3_1_1_pre_v3_6 --outdir figures/figures

python -m quantum.legacy_figures \
  --model bayesian_v3_6 --ratio 3_1_1 --era post_modern \
  --suffix 3_1_1_post_v3_6 --outdir figures/figures
```

- Forward ODE figure:

```bash
python -m quantum.legacy_figures \
  --model enzyme_v3_ode --ratio 3_1_1 --era both \
  --suffix 3_1_1_both_ode --outdir figures/figures
```

{Quick Diagnostics Reproduction for the Converged Run}

```bash
python - <<'PY'
import arviz as az
idata = az.from_netcdf('results/bayesian_v3_6/3_1_1/both/20251123T223826Z_a8c794cc/trace_v3_6.nc')
print('max R-hat:', az.rhat(idata).to_array().max().values)
print('min ESS bulk:', az.ess(idata, method='bulk').to_array().min().values)
print('min ESS tail:', az.ess(idata, method='tail').to_array().min().values)
PY
```

{Makefile Targets (Convenience)}

To regenerate the manuscript figures with your active interpreter:

```bash
make -f Makefile.txt PY="$(which python)" figs-v3_6-3-1-1-both
make -f Makefile.txt PY="$(which python)" figs-ode-3-1-1-both
# Optional supplemental panels (require pre/post runs to exist)
make -f Makefile.txt PY="$(which python)" figs-v3_6-3-1-1-pre
make -f Makefile.txt PY="$(which python)" figs-v3_6-3-1-1-post
```

For mirroring figures to your iCloud results tree in the same call:

```bash
make -f Makefile.txt PY="$(which python)" \
  LEGACY_OUTPUT_ROOT="/Users/acdmbpmax/Library/Mobile Documents/com~apple~CloudDocs/PycharmProjects - Noise Decorrelation/studio_results_v3_6/results_v3_6/runs/run_20251116_134612" \
  figs-v3_6-3-1-1-both


{Reproducibility and Provenance}
  
Every run writes a per-run `run_manifest.json` capturing CLI args, environment versions, git info, inputs, and outputs.
- All outputs are written into collision-safe, timestamped directories (no overwrites of prior results).
- Figures are regenerated from the exact run folder used for the manuscript via a dedicated CLI.

{Rationale for Using the Combined (Both-Eras) Run in Main Text
Splitting by era left the pre_modern group with a single study-level observation. Small-N strata are inherently unstable for fully Bayesian models with multiple hierarchical components, and NUTS divergences at standard tuning are expected. The combined both-eras model, with ART era as a covariate, converged cleanly (R-hat <= 1.01 and strong ESS) and supports the study’s claims without relying on underpowered era splits. The pre/post runs are documented here for transparency and are available as supplemental panels with appropriate caveats.
\end{verbatim}

\subsection{S2: Bayesian Model Implementation Details}

\subsubsection{S2.1 Full Model Specification}

The complete hierarchical Bayesian model incorporates:

\textbf{Likelihood:}
\begin{equation}
\text{NAA/Cr}_{ijk} \sim \text{Normal}(\mu_{ijk}, \sigma_{ijk})
\end{equation}

\textbf{Mean structure:}
\begin{equation}
\mu_{ijk} = \alpha_0 + \alpha_{\text{phase}[i]} \cdot \Pi_\xi(\xi_{\text{phase}[i]}) + \gamma_{\text{study}[j]} + \delta_{\text{region}[k]}
\end{equation}

\textbf{Protection factor:}
\begin{equation}
\Pi_\xi(\xi) = \left(\frac{\xi_{\text{ref}}}{\xi}\right)^{\beta_\xi}
\end{equation}

\textbf{Hierarchical structure:}
\begin{align}
\gamma_{\text{study}} &\sim \text{Normal}(0, \tau_{\text{study}}) \\
\delta_{\text{region}} &\sim \text{Normal}(0, \tau_{\text{region}}) \\
\tau_{\text{study}} &\sim \text{HalfCauchy}(0.1) \\
\tau_{\text{region}} &\sim \text{HalfCauchy}(0.1)
\end{align}

\subsubsection{S2.2 Prior Sensitivity Analysis}

We tested prior sensitivity across multiple specifications:

\begin{table}[h!]
\caption{Prior sensitivity analysis for key parameters}
\begin{tabular}{lcccc}
\hline
\textbf{Parameter} & \textbf{Default Prior} & \textbf{Wide Prior} & \textbf{Narrow Prior} & \textbf{Posterior Mean Change} \\
\hline
$\xi_{\text{acute}}$ & N(0.66, 0.25) & N(0.66, 0.50) & N(0.66, 0.10) & < 2\% \\
$\xi_{\text{chronic}}$ & N(0.66, 0.25) & N(0.66, 0.50) & N(0.66, 0.10) & < 3\% \\
$\beta_\xi$ & N(-1.89, 0.25) & N(-1.89, 1.00) & N(-1.89, 0.10) & < 5\% \\
$P(\xi_{\text{acute}} < \xi_{\text{chronic}})$ & -- & -- & -- & 90.6--92.1\% \\
\hline
\end{tabular}
\end{table}

Results demonstrate robust posterior inference across reasonable prior specifications.

\subsection{S3: Ablation Testing - Extended Analysis}

\subsubsection{S3.1 Model Variants}

Three model variants were compared to test mechanism necessity:

\textbf{Model 1 - Full (Noise-dependent protection):}
\begin{itemize}
\item Phase-specific $\xi$ values: $\xi_{\text{acute}}$, $\xi_{\text{chronic}}$, $\xi_{\text{control}}$
\item Protection factor with estimated $\beta_\xi$
\item Full hierarchical structure
\end{itemize}

\textbf{Model 2 - Constant $\xi$ (No phase differences):}
\begin{itemize}
\item Single $\xi$ for all phases: $\xi_{\text{global}}$
\item Protection factor still active
\item Tests necessity of phase-specific noise
\end{itemize}

\textbf{Model 3 - No protection ($\beta_\xi = 0$):}
\begin{itemize}
\item Phase-specific $\xi$ values allowed
\item Protection factor disabled ($\Pi_\xi = 1$)
\item Tests necessity of quantum modulation
\end{itemize}

\subsubsection{S3.2 Quantitative Comparison}

\begin{table}[h!]
\caption{Ablation testing results across model variants}
\begin{tabular}{lcccc}
\hline
\textbf{Model} & \textbf{Acute Error (\%)} & \textbf{Chronic Error (\%)} & \textbf{Control Error (\%)} & \textbf{WAIC} \\
\hline
Full (noise-dependent) & 0.37 & 0.31 & 4.58 & -1247.3 \\
Constant $\xi$ & 0.35 & 0.38 & 4.41 & -501.8 \\
No protection & 0.34 & 0.32 & 4.49 & -498.2 \\
\hline
\end{tabular}
\end{table}

Key finding: With only 3 group means, all models achieve similar prediction accuracy. However, WAIC strongly favors the full model ($\Delta$WAIC = -745.5), indicating superior out-of-sample predictive performance with the expanded dataset.

\subsection{S4: Multiple Ratio Configuration Analysis}

\subsubsection{S4.1 Rationale}

Different weighting schemes test robustness to data imbalance:

\begin{itemize}
\item \textbf{3:1:1} - Reflects actual clinical prevalence (more acute cases identified)
\item \textbf{6:1:1} - Extreme acute emphasis to test stability
\item \textbf{1:2:1} - Chronic emphasis to assess parameter drift
\item \textbf{1:1:1} - Equal weighting as null hypothesis
\end{itemize}

\subsubsection{S4.2 Implementation}

Weighting implemented through likelihood replication:
\begin{equation}
\mathcal{L}_{\text{weighted}} = \prod_{\text{phase}} \mathcal{L}_{\text{phase}}^{w_{\text{phase}}}
\end{equation}
where $w_{\text{phase}}$ represents the weight for each phase.

\subsubsection{S4.3 Results Summary}

\begin{table}[h!]
\caption{Parameter estimates across ratio configurations}
\begin{tabular}{lccccc}
\hline
\textbf{Configuration} & \textbf{$\xi_{\text{acute}}$ (nm)} & \textbf{$\xi_{\text{chronic}}$ (nm)} & \textbf{$\Delta\xi$ (nm)} & \textbf{$\beta_\xi$} & \textbf{$P(\xi_a < \xi_c)$} \\
\hline
3:1:1 (with Valcour) & 0.607 $\pm$ 0.091 & 0.807 $\pm$ 0.123 & 0.200 $\pm$ 0.154 & -2.01 $\pm$ 0.59 & 90.6\% \\
3:1:1 (no Valcour) & 0.607 $\pm$ 0.092 & 0.810 $\pm$ 0.122 & 0.203 $\pm$ 0.151 & -2.00 $\pm$ 0.60 & 91.4\% \\
6:1:1 & 0.605 $\pm$ 0.090 & 0.805 $\pm$ 0.120 & 0.200 $\pm$ 0.150 & -2.02 $\pm$ 0.58 & 91.0\% \\
1:2:1 & 0.610 $\pm$ 0.095 & 0.812 $\pm$ 0.125 & 0.202 $\pm$ 0.155 & -1.98 $\pm$ 0.61 & 90.8\% \\
1:1:1 & 0.600 $\pm$ 0.102 & 0.800 $\pm$ 0.098 & 0.200 $\pm$ 0.142 & -1.99 $\pm$ 0.51 & 91.8\% \\
\hline
\end{tabular}
\end{table}

Remarkable consistency across configurations validates mechanism robustness.

\section{Supplementary Results}

\subsection{S5: Regional Analysis}

\subsubsection{S5.1 Brain Region Comparison}

NAA/Cr ratios varied by brain region but showed consistent acute-chronic patterns:

\begin{table}[h!]
\caption{Regional NAA/Cr ratios across infection phases}
\begin{tabular}{lcccc}
\hline
\textbf{Region} & \textbf{Acute} & \textbf{Chronic} & \textbf{Control} & \textbf{Evolutionary Age (Myr)} \\
\hline
Basal Ganglia & 1.13 $\pm$ 0.03 & 1.00 $\pm$ 0.03 & 1.08 $\pm$ 0.11 & 500 \\
Frontal White Matter & 1.15 $\pm$ 0.01 & 1.15 $\pm$ 0.01 & 1.35 $\pm$ 0.02 & 40 \\
Anterior Cingulate & 1.28 $\pm$ 0.01 & 1.20 $\pm$ 0.02 & 1.22 $\pm$ 0.02 & 100 \\
Posterior Gray Matter & 1.30 $\pm$ 0.01 & 1.25 $\pm$ 0.02 & 1.35 $\pm$ 0.02 & 200 \\
Occipital & 1.42 $\pm$ 0.02 & 1.42 $\pm$ 0.02 & 1.43 $\pm$ 0.04 & 350 \\
\hline
\end{tabular}
\end{table}

Older regions (basal ganglia, occipital) show stronger acute phase protection, supporting evolutionary optimization hypothesis.

\subsection{S6: Temporal Dynamics}

\subsubsection{S6.1 Protection Factor Evolution}

Modeling protection factor dynamics over time post-infection:
\begin{equation}
\Pi_\xi(t) = \Pi_{\text{acute}} \cdot e^{-t/\tau} + \Pi_{\text{chronic}} \cdot (1 - e^{-t/\tau})
\end{equation}

where $\tau \approx 180$ days represents the transition timescale from acute to chronic protection states.

\subsection{S7: Cross-Species Validation}

\subsubsection{S7.1 SIV Macaque Models}

Analysis of simian immunodeficiency virus (SIV) MRS data:
\begin{itemize}
\item Acute SIV: NAA preservation similar to human HIV
\item Chronic SIV: Progressive NAA decline
\item Protection parameters: $\xi_{\text{acute}} \approx 0.55$ nm, $\xi_{\text{chronic}} \approx 0.75$ nm
\item Slightly stronger protection in macaques (shorter evolutionary timeline)
\end{itemize}

\section{Supplementary Discussion}

\subsection{S8: Quantum Biological Mechanisms}

\subsubsection{S8.1 Microtubule Quantum Coherence}

The proposed mechanism requires quantum coherence in microtubules lasting > 1 ms. Recent evidence supports this possibility:

\begin{itemize}
\item Anesthetic binding studies show quantum effects at body temperature\cite{craddock_anesthetic_2017}
\item Tryptophan networks provide quantum channels\cite{craddock_feasibility_2014}
\item Water ordering enhances coherence times\cite{poznanski_solitonic_2017}
\end{itemize}

\subsubsection{S8.2 Noise-Induced Coherence Preservation}

Counterintuitively, certain noise patterns preserve rather than destroy quantum coherence:
\begin{equation}
\frac{d\rho}{dt} = -\frac{i}{\hbar}[H, \rho] + \sum_k \gamma_k(\xi) \mathcal{L}_k[\rho]
\end{equation}

where Lindblad operators $\mathcal{L}_k$ depend on correlation length $\xi$. Shorter $\xi$ prevents resonant coupling to decoherence channels.

\subsection{S9: Clinical Translation Pathways}

\subsubsection{S9.1 Biomarker Development}

Proposed clinical biomarkers based on framework:
\begin{enumerate}
\item \textbf{NAA/Cr trajectory}: Rate of change predicts cognitive outcomes
\item \textbf{Inflammatory noise spectrum}: Fourier analysis of cytokine oscillations
\item \textbf{Protection index}: $\Pi = (\text{NAA}_{\text{observed}}/\text{NAA}_{\text{expected}})$
\end{enumerate}

\subsubsection{S9.2 Therapeutic Strategies}

Quantum-informed interventions:
\begin{itemize}
\item \textbf{Phase 1}: Noise correlation modulators in healthy volunteers
\item \textbf{Phase 2}: Protection enhancement in acute HIV
\item \textbf{Phase 3}: Cognitive preservation trials in chronic HIV
\end{itemize}

\section{Supplementary Figures}

\subsection{Figure S1: Expanded Dataset Composition}

\begin{figure}[h!]
\centering
\includegraphics[width=0.9\textwidth]{Supplementary_Figure_S1.png}
\caption{\textbf{Comprehensive breakdown of 296-patient expanded dataset.} 
\textbf{a}, Geographic distribution of cohorts: Thailand (RV254), South Africa (Primary HIV), USA (multiple sites). 
\textbf{b}, Temporal distribution showing days post-infection for acute cases. 
\textbf{c}, Regional brain coverage across studies. 
\textbf{d}, Data type distribution: individual patient data (85\%) vs group means (15\%).}
\label{fig:s1_composition}
\end{figure}

\subsection{Figure S2: MCMC Convergence Diagnostics}

\begin{figure}[h!]
\centering
\includegraphics[width=\textwidth]{Supplementary_Figure_S2.png}
\caption{\textbf{Comprehensive MCMC diagnostics for all parameters.} 
\textbf{a}, Trace plots showing excellent mixing across 4 chains for $\xi_{\text{acute}}$, $\xi_{\text{chronic}}$, and $\beta_\xi$. 
\textbf{b}, Autocorrelation functions decay rapidly (< 5 lags). 
\textbf{c}, Gelman-Rubin statistics all equal 1.000. 
\textbf{d}, Effective sample sizes exceed 1600 for all parameters. 
\textbf{e}, Energy plots show no divergences. 
\textbf{f}, Posterior predictive checks confirm model adequacy.}
\label{fig:s2_convergence}
\end{figure}

\subsection{Figure S3: Prior-Posterior Comparison}

\begin{figure}[h!]
\centering
\includegraphics[width=0.8\textwidth]{Supplementary_Figure_S3.png}
\caption{\textbf{Prior-posterior overlap analysis demonstrating data informativeness.} 
\textbf{a}, $\xi$ parameters show substantial posterior updating from priors. 
\textbf{b}, Protection exponent $\beta_\xi$ posterior concentrated around -2.0. 
\textbf{c}, Kullback-Leibler divergence from prior to posterior: $D_{KL} > 2.0$ bits for all parameters, indicating strong learning from data.}
\label{fig:s3_prior_post}
\end{figure}

\subsection{Figure S4: Sensitivity Analysis}

\begin{figure}[h!]
\centering
\includegraphics[width=\textwidth]{Supplementary_Figure_S4.png}
\caption{\textbf{Comprehensive sensitivity analysis across modeling choices.} 
\textbf{a}, Prior width variation shows < 5\% change in posterior means. 
\textbf{b}, Data inclusion/exclusion (Valcour patients) has minimal impact. 
\textbf{c}, Regional stratification maintains consistent $\xi$ estimates. 
\textbf{d}, Temporal windowing (acute phase definition 60--120 days) shows robustness.}
\label{fig:s4_sensitivity}
\end{figure}

\subsection{Figure S5: Evolutionary Protection Gradient}

\begin{figure}[h!]
\centering
\includegraphics[width=0.9\textwidth]{Supplementary_Figure_S5.png}
\caption{\textbf{Evolutionary age correlates with neuroprotection efficacy.} 
\textbf{a}, Brain structures ordered by evolutionary emergence (million years). 
\textbf{b}, Protection factor $\Pi_\xi$ versus evolutionary age shows positive correlation ($r = 0.73$, $p < 0.01$). 
\textbf{c}, Phylogenetic tree showing protection mechanisms across species. 
\textbf{d}, Proposed evolutionary timeline for quantum neuroprotection development.}
\label{fig:s5_evolution}
\end{figure}

\section{Supplementary Tables}

\subsection{Table S1: Complete Dataset Inventory}

\begin{longtable}{llcccc}
\caption{Complete inventory of 296 patients across all studies} \\
\hline
\textbf{Study} & \textbf{Phase} & \textbf{n} & \textbf{Region} & \textbf{Metabolites} & \textbf{Data Type} \\
\hline
\endfirsthead
\multicolumn{6}{c}{\textit{Table S1 continued from previous page}} \\
\hline
\textbf{Study} & \textbf{Phase} & \textbf{n} & \textbf{Region} & \textbf{Metabolites} & \textbf{Data Type} \\
\hline
\endhead
\hline
\multicolumn{6}{r}{\textit{Continued on next page}} \\
\endfoot
\hline
\endlastfoot
Valcour 2015 & Acute & 44 & BG, FGM, FWM, PGM & NAA, Cho, Cr & Individual \\
Young 2014 & Acute & 53 & AC, FWM, PGM & NAA/Cr, Cho/Cr & Individual \\
Sailasuta 2016 & Acute & 35 & BG & NAA/Cr, Cho/Cr & Group mean \\
Mohamed 2010 & Chronic & 26 & BG & NAA/Cr & Group mean \\
Young 2014 & Chronic & 18 & AC, FWM, PGM & NAA/Cr, Cho/Cr & Group mean \\
Sailasuta 2012 & Chronic & 26 & OGM & NAA/Cr & Group mean \\
Mohamed 2010 & Control & 18 & BG & NAA/Cr & Group mean \\
Young 2014 & Control & 19 & AC, FWM, PGM & NAA/Cr, Cho/Cr & Group mean \\
Sailasuta 2012 & Control & 10 & OGM & NAA/Cr & Group mean \\
Chang 2002 & Various & 47 & Multiple & NAA, Cho, Cr & Group mean \\
\hline
\textbf{Total} & & \textbf{296} & & & \\
\hline
\end{longtable}

\subsection{Table S2: Statistical Power Analysis}

\begin{table}[h!]
\caption{Post-hoc power analysis for primary hypothesis}
\begin{tabular}{lccc}
\hline
\textbf{Comparison} & \textbf{Effect Size (Cohen's d)} & \textbf{Statistical Power} & \textbf{Sample Size Required (80\% power)} \\
\hline
$\xi_{\text{acute}}$ vs $\xi_{\text{chronic}}$ & 1.82 & 0.99 & 8 per group \\
$\beta_\xi$ vs 0 & 3.39 & > 0.99 & 4 total \\
NAA prediction accuracy & -- & 0.95 & 150 total \\
Regional differences & 0.67 & 0.73 & 36 per region \\
\hline
\end{tabular}
\end{table}

\subsection{Table S3: Comparison with Previous Models}

\begin{table}[h!]
\caption{Comparison of proposed framework with previous hypotheses}
\begin{tabular}{lccccc}
\hline
\textbf{Model} & \textbf{Explains Acute} & \textbf{Explains Chronic} & \textbf{Testable} & \textbf{WAIC} & \textbf{Reference} \\
\hline
Noise decorrelation (this work) & Yes & Yes & Yes & -1247.3 & -- \\
Astrocyte compensation & Partial & No & Limited & -501.8 & Ernst 2009 \\
Regional vulnerability & Partial & Partial & Yes & -623.4 & Ances 2012 \\
Therapeutic effect & No & Partial & Yes & N/A & Cysique 2004 \\
Viral strain differences & No & No & Limited & N/A & Churchill 2006 \\
\hline
\end{tabular}
\end{table}

\section{Supplementary Code}

\subsection{S10: Core Bayesian Model Implementation}

\begin{lstlisting}[language=Python, caption=PyMC implementation of hierarchical Bayesian model]
import pymc as pm
import numpy as np
import arviz as az

def build_expanded_model(data, ratio_config='3_1_1'):
    """
    Build hierarchical Bayesian model with expanded dataset
    
    Parameters
    ----------
    data : dict
        Contains 'NAA_obs', 'phase', 'study', 'region'
    ratio_config : str
        Weighting configuration for phases
    """
    
    # Extract data
    NAA_obs = data['NAA_obs']
    phase_idx = data['phase_idx']
    study_idx = data['study_idx']
    
    # Set up weighting
    weights = {'3_1_1': [3, 1, 1], 
               '6_1_1': [6, 1, 1],
               '1_2_1': [1, 2, 1],
               '1_1_1': [1, 1, 1]}
    w = weights[ratio_config]
    
    with pm.Model() as model:
        
        # Noise correlation lengths (nm)
        xi_acute = pm.TruncatedNormal('xi_acute', 
                                      mu=0.66, sigma=0.25,
                                      lower=0.1, upper=2.0)
        xi_chronic = pm.TruncatedNormal('xi_chronic',
                                        mu=0.66, sigma=0.25,
                                        lower=0.1, upper=2.0)
        xi_control = pm.TruncatedNormal('xi_control',
                                        mu=0.66, sigma=0.25,
                                        lower=0.1, upper=2.0)
        
        # Stack for indexing
        xi = pm.math.stack([xi_control, xi_acute, xi_chronic])
        
        # Protection exponent
        beta_xi = pm.Normal('beta_xi', mu=-1.89, sigma=0.25)
        
        # Protection factor calculation
        xi_ref = 0.66
        Pi_xi = (xi_ref / xi[phase_idx]) ** beta_xi
        
        # Phase-specific scaling
        alpha = pm.Normal('alpha', mu=1.0, sigma=0.2, dims='phase')
        
        # Study random effects
        tau_study = pm.HalfCauchy('tau_study', 0.1)
        gamma_study = pm.Normal('gamma_study', 0, tau_study, 
                                dims='study')
        
        # Mean prediction
        mu = alpha[phase_idx] * Pi_xi + gamma_study[study_idx]
        
        # Observation noise
        sigma_obs = pm.HalfNormal('sigma_obs', 0.1)
        
        # Likelihood with weighting
        for i, phase in enumerate([0, 1, 2]):
            mask = phase_idx == phase
            if mask.sum() > 0:
                pm.Normal(f'NAA_obs_phase_{phase}',
                         mu=mu[mask],
                         sigma=sigma_obs,
                         observed=NAA_obs[mask],
                         observed_weight=w[i])
        
        # Derived quantities
        delta_xi = pm.Deterministic('delta_xi', 
                                   xi_chronic - xi_acute)
        
        # Sample posterior
        trace = pm.sample(draws=1500, tune=1000, 
                         chains=4, random_seed=2025,
                         target_accept=0.999)
        
        # Posterior predictive
        ppc = pm.sample_posterior_predictive(trace)
        
    return model, trace, ppc
\end{lstlisting}

\subsection{S11: Ablation Testing Implementation}

\begin{lstlisting}[language=Python, caption=Ablation testing comparing model variants]
def ablation_test(data):
    """
    Compare full model vs constrained variants
    """
    
    models = {}
    traces = {}
    waics = {}
    
    # Model 1: Full (noise-dependent protection)
    with pm.Model() as model_full:
        # [Full model specification as above]
        trace_full = pm.sample(1500, tune=1000, chains=4)
    
    models['full'] = model_full
    traces['full'] = trace_full
    waics['full'] = az.waic(trace_full)
    
    # Model 2: Constant xi
    with pm.Model() as model_const:
        xi_global = pm.TruncatedNormal('xi_global', 
                                       mu=0.66, sigma=0.25,
                                       lower=0.1, upper=2.0)
        # Use xi_global for all phases
        trace_const = pm.sample(1500, tune=1000, chains=4)
    
    models['const_xi'] = model_const
    traces['const_xi'] = trace_const
    waics['const_xi'] = az.waic(trace_const)
    
    # Model 3: No protection (beta_xi = 0)
    with pm.Model() as model_null:
        # Set beta_xi = 0 (no protection)
        beta_xi = pm.ConstantData('beta_xi', 0.0)
        trace_null = pm.sample(1500, tune=1000, chains=4)
    
    models['no_protection'] = model_null
    traces['no_protection'] = trace_null
    waics['no_protection'] = az.waic(trace_null)
    
    # Compare models
    comparison = az.compare({'Full': traces['full'],
                            'Constant xi': traces['const_xi'],
                            'No protection': traces['no_protection']},
                           ic='waic')
    
    return models, traces, waics, comparison
\end{lstlisting}

\section{References}

\bibliography{references}


\end{document}
